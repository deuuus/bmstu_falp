\documentclass[12pt]{report}
\usepackage[utf8]{inputenc}
\usepackage[russian]{babel}
%\usepackage[14pt]{extsizes}
\usepackage{listings}
\usepackage{graphicx}
\usepackage{amsmath,amsfonts,amssymb,amsthm,mathtools} 
\usepackage{pgfplots}
\usepackage{filecontents}
\usepackage{indentfirst}
\usepackage{eucal}
\usepackage{float} 
\usepackage{amsmath}
\usepackage{enumitem}
\usepackage[justification=centering]{caption} 
\usepackage{tikz}
\usepackage{pgfplots}
\pgfplotsset{compat=newest}

\frenchspacing

\usepackage{indentfirst} % Красная строка


%\usetikzlibrary{datavisualization}
%\usetikzlibrary{datavisualization.formats.functions}

\usepackage{amsmath}


% Для листинга кода:
\lstset{ %
	language=haskell,                 % выбор языка для подсветки (здесь это С)
	basicstyle=\small\sffamily, % размер и начертание шрифта для подсветки кода
	numbers=left,               % где поставить нумерацию строк (слева\справа)
	numberstyle=\tiny,           % размер шрифта для номеров строк
	stepnumber=1,                   % размер шага между двумя номерами строк
	numbersep=5pt,                % как далеко отстоят номера строк от подсвечиваемого кода
	showspaces=false,            % показывать или нет пробелы специальными отступами
	showstringspaces=false,      % показывать или нет пробелы в строках
	showtabs=false,             % показывать или нет табуляцию в строках
	frame=single,              % рисовать рамку вокруг кода
	tabsize=2,                 % размер табуляции по умолчанию равен 2 пробелам
	captionpos=t,              % позиция заголовка вверху [t] или внизу [b] 
	breaklines=true,           % автоматически переносить строки (да\нет)
	breakatwhitespace=false, % переносить строки только если есть пробел
	escapeinside={\#*}{*)}   % если нужно добавить комментарии в коде
}

\usepackage[left=2cm,right=2cm, top=2cm,bottom=2cm,bindingoffset=0cm]{geometry}

\usepackage{listings}

\usepackage{titlesec}
\titleformat{\section}
{\normalsize\bfseries}
{\thesection}
{1em}{}
\titlespacing*{\chapter}{0pt}{-30pt}{8pt}
\titlespacing*{\section}{\parindent}{*4}{*4}
\titlespacing*{\subsection}{\parindent}{*4}{*4}
\usepackage{setspace}

\titleformat{\chapter}{\LARGE\bfseries}{\thechapter}{20pt}{\large\bfseries}
\titleformat{\section}{\Large\bfseries}{\thesection}{20pt}{\large\bfseries}

\makeatletter 

\begin{document}
	
%\def\chaptername{} % убирает "Глава"
\thispagestyle{empty}
\begin{titlepage}
	\noindent \begin{minipage}{0.15\textwidth}
		\includegraphics[width=\linewidth]{pics/logo}
	\end{minipage}
	\noindent\begin{minipage}{0.9\textwidth}\centering
		\textbf{Министерство науки и высшего образования Российской Федерации}\\
		\textbf{Федеральное государственное бюджетное образовательное учреждение высшего образования}\\
		\textbf{~~~«Московский государственный технический университет имени Н.Э.~Баумана}\\
		\textbf{(национальный исследовательский университет)»}\\
		\textbf{(МГТУ им. Н.Э.~Баумана)}
	\end{minipage}
	
	\noindent\rule{18cm}{3pt}
	\newline\newline
	\noindent ФАКУЛЬТЕТ $\underline{\text{«Информатика и системы управления»}}$ \newline\newline
	\noindent КАФЕДРА $\underline{\text{«Программное обеспечение ЭВМ и информационные технологии»}}$\newline\newline\newline\newline\newline
	
	
	\begin{center}
		\noindent\begin{minipage}{1.3\textwidth}\centering
			\Large\textbf{  Отчёт по лабораторной работе №4 по курсу}\newline\newline
			\textbf{<<Функциональное и логическое}\newline
			\textbf{\indent\indent\indent программирование>>}\newline
		\end{minipage}
	\end{center}
	
	~\\\\\\\\\\\\
	\large
	\noindent\textbf{Тема } $\underline{\text{Использование управляющих структур, работа со списками.}}$\newline\newline
	\noindent\textbf{Студент } $\underline{\text{Сироткина П.Ю.}}$\newline\newline
	\noindent\textbf{Группа } $\underline{\text{ИУ7-66Б}}$\newline\newline
	\noindent\textbf{Преподаватели } $\underline{\text{Толпинская Н.Б., Строганов Ю.В.}}$\newline\newline\newline
	
	\begin{center}
		\vfill
		Москва~---~\the\year
		~г.
	\end{center}
\end{titlepage}

\chapter{Практические задания}
%\addcontentsline{toc}{chapter}{Введение}

\section{Чем принципиально отличаются функции cons, list, append? Пусть (setf lst1 '(a b)) (setf lst2 '(c d)). Каковы результаты вычисления следующих выражений?}

\begin{lstlisting}
	(setf lst1 '(a b))
	(setf lst2 '(c d))
	(cons lst1 lst2)      ; ((A B) C D)
	(list lst1 lst2)      ; ((A B) (C D))
	(append lst1 lst2)    ; (A B C D)
\end{lstlisting}

\section{Каковы результаты вычисления следующих выражений, и почему?}

\begin{lstlisting}
	(reverse ())          ; NIL
	(last ())             ; NIL
	(reverse '(a))        ; (A)
	(last '(a))           ; (A)
	(reverse '((a b c)))  ; ((A B C))
	(last '((a b c)))     ; ((A B C))
\end{lstlisting}

\section{Написать по крайней мере два варианта функции, которая возвращает последний элемент своего списка-аргумента.}

\begin{lstlisting}
	(defun get_last(lst) (last lst))
\end{lstlisting}

\begin{lstlisting}
	(defun get_last(lst) (car (reverse lst)))
\end{lstlisting}

\begin{lstlisting}
	(defun get_last(lst) (if (cdr lst) (get_last(cdr lst)) (car lst)))
\end{lstlisting}

\section{Написать по крайней мере два варианта функции, которая возвращает свой список-аргумент без последнего элемента.}

\begin{lstlisting}
	(defun del_last(lst) (reverse (cdr (reverse lst))))
\end{lstlisting}

\begin{lstlisting}
	(defun del_last(lst) (if (cdr lst)) (cons (car lst) (del_last (cdr lst))) Nil)
\end{lstlisting}

\section{Написать простой вариант игры в кости.}

Написать простой вариант игры в кости, в котором бросаются две правильные кости. Если сумма выпавших очков равна 7 или 11 -- выигрыш, если выпало (1, 1) или (6, 6) -- игрок в праве снова бросить кости, во всех остальных случаях ход переходит ко второму игроку, но запоминается сумма выпавших очков. Если второй игрок не выигрывает абсолютно, то выигрывает тот игрок, у которого больше очков. Результат игры и значения выпавших костей выводить на экран с помощью функции print.

\begin{lstlisting}
(defun roll_dices()
(list (+ (random 6) 1) (+ (random 6) 1)))

(defun sum_points(dices)
(+ (first dices) (second dices)))

(defun is_absolute_winner(dices)
(or (= (sum_points dices) 7) (= (sum_points dices) 11)))

(defun need_reroll(dices)
(or (= (sum_points dices) 2) (= (sum_points dices) 12)))

(defun process(fdices sdices)
(setq fsum (sum_points fdices))
(setq ssum (sum_points sdices))

(if (need_reroll fdices) (print (and (print 'Reroll_for_player_#1) (setq fdices (roll_dices)))))

(if (is_absolute_winner fdices) 
'Player_#1_win
(if (is_absolute_winner sdices) 
'Player_#2_win
(if (> (sum_points sdices) (sum_points fdices)) 'Player_#2_win 'Player_#1_win)
)
)
)

(defun play()
(print '=========Roll_dices=========)
(setq fdices (roll_dices))
(setq sdices (roll_dices))

(print 'Player_#1) 
(print fdices)
(terpri)

(print 'Player_#2) 
(print sdices)
(terpri)

(print '=========Start_game=========)
(print (process fdices sdices))
(terpri)
)

(play)
\end{lstlisting}

\end{document}