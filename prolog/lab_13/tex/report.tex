\documentclass[12pt,a4paper]{scrreprt}

\usepackage{cmap}
\usepackage[T1]{fontenc} 
\usepackage[utf8]{inputenc}
\usepackage[english,russian]{babel}

\usepackage{caption}
\usepackage{subcaption}
\usepackage[normalem]{ulem}

%\usepackage{soulutf8}

\usepackage{float}

\usepackage{enumitem}

\usepackage{graphicx}
\usepackage{multirow}


\usepackage{pgfplots}
\pgfplotsset{compat=newest}
\usepgfplotslibrary{units}


\usepackage{caption}
\captionsetup{labelsep=endash}
\captionsetup[figure]{name={Рисунок}}
\captionsetup[subfigure]{name={Рисунок}}
\captionsetup[subtable]{labelformat=simple}
\captionsetup[subfigure]{labelformat=simple}
\renewcommand{\thesubtable}{\text{Таблица }\arabic{chapter}\text{.}\arabic{table}\text{.}\arabic{subtable}\text{ --}}
%\renewcommand{\thesubfigure}{\arabic{chapter}\text{.}\arabic{figure}\text{.}\azbuk{\subfigure}\text{ --}}


\usepackage{textcomp}
\usepackage{chngcntr}

\usepackage{amsmath}
\usepackage{amsfonts}
\usepackage{array}

\usepackage{geometry}
\geometry{left=20mm}
\geometry{right=10mm}
\geometry{top=20mm}
\geometry{bottom=20mm}
\geometry{foot=1.7cm}

\usepackage{listings}
\lstset{ %
	language=haskell,                 % выбор языка для подсветки (здесь это С)
	basicstyle=\small\sffamily, % размер и начертание шрифта для подсветки кода
	numbers=left,               % где поставить нумерацию строк (слева\справа)
	numberstyle=\tiny,           % размер шрифта для номеров строк
	stepnumber=1,                   % размер шага между двумя номерами строк
	numbersep=5pt,                % как далеко отстоят номера строк от подсвечиваемого кода
	showspaces=false,            % показывать или нет пробелы специальными отступами
	showstringspaces=false,      % показывать или нет пробелы в строках
	showtabs=false,             % показывать или нет табуляцию в строках
	frame=single,              % рисовать рамку вокруг кода
	tabsize=2,                 % размер табуляции по умолчанию равен 2 пробелам
	captionpos=t,              % позиция заголовка вверху [t] или внизу [b] 
	breaklines=true,           % автоматически переносить строки (да\нет)
	breakatwhitespace=false, % переносить строки только если есть пробел
	escapeinside={\#*}{*)}   % если нужно добавить комментарии в коде
}


\usepackage{titlesec}
\titleformat{\section}
{\normalsize\bfseries}
{\thesection}
{1em}{}
\titlespacing*{\chapter}{0pt}{-30pt}{8pt}
\titlespacing*{\section}{\parindent}{*4}{*4}
\titlespacing*{\subsection}{\parindent}{*4}{*4}
\titlespacing*{\subsubsection}{\parindent}{*4}{*4}

%\renewcommand\thesubfloat{(\roman{subfloat})}
\renewcommand\thesubfigure{(\asbuk{subfigure})}

% Маркировка для списков
\def\labelitemi{$\circ$}
\def\labelitemii{$*$}
\usepackage{pdflscape}

\usepackage{setspace}
\onehalfspacing % Полуторный интервал

\captionsetup[table]{skip=0pt,singlelinecheck=off, justification=raggedleft}
\captionsetup[table]{skip=0pt,singlelinecheck=off, justification=centering}

\frenchspacing
\usepackage{indentfirst} % Красная строка

\usepackage{titlesec}
\usepackage{xcolor}
% Названия глав
\titleformat{\section}{\huge\textmd}{\thesection}{1em}{}

\definecolor{gray35}{gray}{0.35}

\titleformat{\chapter}[hang]{\Huge}{\textcolor{gray35}{\thechapter. }}{0pt}{\huge\scshape}

\titleformat{\section}{\Large}{\textcolor{gray35}\thesection}{20pt}{\Large\scshape}
\titleformat{\subsection}{\large}{\thesubsection}{20pt}{\large\scshape}
\titleformat{\subsubsection}{\large}{\thesubsubsection}{20pt}{\large\scshape}

\newcommand*{\undertext}[2]{%
	\begin{tabular}[t]{@{}c@{}}%
		#1\\\relax\scriptsize(#2)%
	\end{tabular}
}

\emergencystretch 10em

% Настройки введения

\addtocontents{toc}{\setcounter{tocdepth}{3}}
\addtocontents{toc}{\setcounter{secnumdepth}{3}}

\usepackage{tocloft,lipsum,pgffor}

\addtocontents{toc}{~\hfill\textnormal{Страница}\par}

\renewcommand{\cftpartfont}{\normalfont\textmd}

\addto\captionsrussian{\renewcommand{\contentsname}{Содержание}}
\renewcommand{\cfttoctitlefont}{\Huge\textmd}

\renewcommand{\cftchapfont}{\normalfont\normalsize}
\renewcommand{\cftsecfont}{\normalfont\normalsize}
\renewcommand{\cftsubsecfont}{\normalfont\normalsize}
\renewcommand{\cftsubsubsecfont}{\normalfont\normalsize}

\renewcommand{\cftchapleader}{\cftdotfill{\cftdotsep}}

\usepackage{listings}
\usepackage{pdflscape}
\usepackage{everypage}
\usepackage{xcolor}

%\bibliographystyle{gost780u.bst}
%
%\usepackage[backend=biber,
%%			bibencoding=utf8,
%%			sorting=nyt,
%%			maxcitenames=2,
%			style=gost-numeric-min,
%%			autolang=other, 
%%			natbib=true,
%%			maxnames=99,
%			uniquename=false]{biblatex}

\usepackage{csquotes} 
\usepackage[backend=biber,
			style=gost-numeric,
			maxcitenames=3,
			maxbibnames=12,
			minnames=1,
			movenames=false,
			ibidtracker=false,
			sorting=none,
			autolang=other]{biblatex}
			
\DeclareSourcemap{
	\maps[datatype=bibtex]{
		\map{
			\step[fieldsource=langid, match=russian, final]
			\step[fieldset=presort, fieldvalue={a}]
		}
		\map{
			\step[fieldsource=langid, notmatch=russian, final]
			\step[fieldset=presort, fieldvalue={z}]
		}
	}
}

\addbibresource{ref-lib.bib} % База библиографии

\usepackage[pdftex]{hyperref} % Гиперссылки
\hypersetup{hidelinks}

% Листинги 
\usepackage{listings}

\definecolor{darkgray}{gray}{0.15}

\definecolor{teal}{rgb}{0.25,0.88,0.73}
\definecolor{gray}{rgb}{0.5,0.5,0.5}
\definecolor{b-red}{rgb}{0.88,0.25,0.41}
\definecolor{royal-blue}{rgb}{0.25,0.41,0.88}



% какой то сложный кусок со стак эксчейндж для квадратных скобок
\makeatletter
\newenvironment{sqcases}{%
	\matrix@check\sqcases\env@sqcases
}{%
	\endarray\right.%
}
\def\env@sqcases{%
	\let\@ifnextchar\new@ifnextchar
	\left\lbrack
	\def\arraystretch{1.2}%
	\array{@{}l@{\quad}l@{}}%
}
\makeatother

% и для матриц
\makeatletter
\renewcommand*\env@matrix[1][\arraystretch]{%
	\edef\arraystretch{#1}%
	\hskip -\arraycolsep
	\let\@ifnextchar\new@ifnextchar
	\array{*\c@MaxMatrixCols c}}
\makeatother

\usepackage{pdflscape}
\usepackage{fancyhdr} 

\fancypagestyle{mylandscape}{
	\fancyhf{} %Clears the header/footer
	\fancyfoot{% Footer
		\makebox[\textwidth][r]{% Right
			\rlap{\hspace{.75cm}% Push out of margin by \footskip
				\smash{% Remove vertical height
					\raisebox{6in}{% Raise vertically
						\rotatebox{90}{\thepage}}}}}}% Rotate counter-clockwise
	\renewcommand{\headrulewidth}{0pt}% No header rule
	\renewcommand{\footrulewidth}{0pt}% No footer rule
}


\begin{document}
	
%\def\chaptername{} % убирает "Глава"
\thispagestyle{empty}
\begin{titlepage}
	\normalsize
	\noindent \begin{minipage}{0.15\textwidth}
		\includegraphics[width=\linewidth]{pics/logo}
	\end{minipage}
	\noindent\begin{minipage}{0.9\textwidth}\centering
		\textbf{Министерство науки и высшего образования Российской Федерации}\\
		\textbf{Федеральное государственное бюджетное образовательное учреждение высшего образования}\\
		\textbf{~~~«Московский государственный технический университет имени Н.Э.~Баумана}\\
		\textbf{(национальный исследовательский университет)»}\\
		\textbf{(МГТУ им. Н.Э.~Баумана)}
	\end{minipage}
	
	\noindent\rule{17cm}{3pt}
	\newline
	\noindent ФАКУЛЬТЕТ $\underline{\text{«Информатика и системы управления»}}$ \newline
	\noindent КАФЕДРА $\underline{\text{«Программное обеспечение ЭВМ и информационные технологии»}}$\newline\newline\newline\newline\newline
	
	\begin{center}
		\noindent\begin{minipage}{1.3\textwidth}\centering
			\Large\textbf{  Отчёт по лабораторной работе №13}\newline
			\textbf{по курсу}\newline
			\textbf{<<Функциональное и логическое}\newline
			\textbf{\indent\indent\indent программирование>>}\newline
		\end{minipage}
	\end{center}
	
	~\\\\\\\\\\\\\\
	\normalsize
	\noindent\textbf{Тема } $\underline{\text{Структура программы на Prolog и ее реализация}}$\newline\newline
	\noindent\textbf{Студент } $\underline{\text{Сироткина П.Ю.}}$\newline\newline
	\noindent\textbf{Группа } $\underline{\text{ИУ7-66Б}}$\newline\newline
	\noindent\textbf{Преподаватели } $\underline{\text{Толпинская Н.Б., Строганов Ю.В.}}$\newline
	
	\begin{center}
		\vfill
		Москва~---~\the\year
		~г.
	\end{center}
\end{titlepage}

\chapter*{Лабораторная работа №13}

Создать базу знаний «Собственники», дополнив (и минимально изменив) базу
знаний, хранящую знания (лаб. 12):

\begin{itemize}
	\item \textbf{«Телефонный справочник»}: Фамилия, № телефона, Адрес – структура (Город,
	Улица, № дома, № кв);
	\item \textbf{«Автомобили»}: Фамилия\_владельца, Марка, Цвет, Стоимость, и др.;
	\item \textbf{«Вкладчики банков»}: Фамилия, Банк, счет, сумма, др.,
	знаниями о дополнительной собственности владельца. Преобразовать знания об
	автомобиле к форме знаний о собственности.
\end{itemize}

Вид собственности (кроме автомобиля):

\begin{itemize}
	\item \textbf{Строение}, стоимость и другие его характеристики;
	\item \textbf{Участок}, стоимость и другие его характеристики;
	\item \textbf{Водный\_транспорт}, стоимость и другие его характеристики.
\end{itemize}

Описать и использовать вариантный домен: \textbf{Собственность}. Владелец может иметь,
но только один объект каждого вида собственности (это касается и автомобиля), или не
иметь некоторых видов собственности.

Используя конъюнктивное правило и разные формы задания одного вопроса (пояснять для какого № задания – какой вопрос), обеспечить возможность поиска:

\begin{enumerate}
	\item Названий всех объектов собственности заданного субъекта.
	\item Названий и стоимости всех объектов собственности заданного субъекта.
	\item * Разработать правило, позволяющее найти суммарную стоимость всех
	объектов собственности заданного субъекта.
\end{enumerate}

Для 2-го пункта и одной фамилии Nсоставить таблицу, отражающую конкретный порядок работы системы, с объяснениями порядка работы и особенностей использования доменов (указать конкретные Т1 и Т2 и полную подстановку на каждом шаге).

\clearpage
\begin{lstlisting}
domains
		surname, phone = symbol.
		city, street = symbol.
		house, flat = integer.
		address = address_struct(city, street, house, flat)
		mark, color = symbol.
		cost = integer.
		bank = symbol.
		account, sum = integer.
		
		area = integer.
		
		property = building(cost, address); 
		land(cost, area); 
		water_transport(cost, mark, color); 
		car(cost, mark, color).

predicates
		has_phone(surname, phone, address).
		has_deposite(surname, bank, account, sum).
		
		own(surname, property).
		
		person_property(surname, symbol).
		person_property_and_cost(surname, symbol, cost).
		
		sum(surname, cost).
		if_has_property(surname, symbol, cost).

clauses
		has_phone(petrov, "111", address_struct(moscow, lenina, 1, 1)).
		has_phone(ivanov, "222", address_struct(vladimir, lomonosova, 2, 2)).
		has_phone(fedorov, "333", address_struct(tver, gagarina, 3, 3)).
		
		has_deposite(petrov, alpha, 0, 1000).
		has_deposite(ivanov, beta, 1, 2000).
		has_deposite(fedorov, gamma, 2, 1500).
		
		own(petrov, building(1000, address_struct(moscow, lenina, 1, 1))).
		own(petrov, water_transport(300, aqua, purple)).
		
		own(ivanov, land(2000, 40)).
		own(ivanov, car(300, bmw, black)).
		
		own(fedorov, building(2000, address_struct(tver, gagarina, 3, 3))).
		own(fedorov, land(500, 25)).
		own(fedorov, water_transport(200, aqua, white)).
		own(fedorov, car(1500, audi, white)).
		
		person_property(Surname, building) :- own(Surname, building(_, _)).
		person_property(Surname, land) :- own(Surname, land(_, _)).
		person_property(Surname, water_transport) :- own(Surname, water_transport(_, _, _)).
		person_property(Surname, car) :- own(Surname, car(_, _, _)).
		
		person_property_and_cost(Surname, building, Cost) :- own(Surname, building(Cost, _)).
		person_property_and_cost(Surname, land, Cost) :- own(Surname, land(Cost, _)).
		person_property_and_cost(Surname, water_transport, Cost) :- own(Surname, water_transport(Cost, _, _)).
		person_property_and_cost(Surname, car, Cost) :- own(Surname, car(Cost, _, _)).
		
		if_has_property(Surname, building, Cost) :- own(Surname, building(Cost, _)), !.
		if_has_property(Surname, land, Cost) :- own(Surname, land(Cost, _)), !.
		if_has_property(Surname, water_transport, Cost) :- own(Surname, water_transport(Cost, _, _)), !.
		if_has_property(Surname, car, Cost) :- own(Surname, car(Cost, _, _)), !.
		if_has_property(_, _, 0).
		
		sum(Surname, Total_sum) :- if_has_property(Surname, building, Building_cost), 
		if_has_property(Surname, land, Land_cost),
		if_has_property(Surname, water_transport, Water_transport_cost),
		if_has_property(Surname, car, Car_cost),
		Total_sum = Building_cost + Land_cost + Water_transport_cost + Car_cost.

goal
		%person_property(petrov, Object).
		%persin_property_and_cost(petrov, Object, Cost).
		%sum("Petrov", Total_sum).
		person_property_and_cost(ivanov, Object, Cost).
\end{lstlisting}

%\clearpage
%\begin{figure}[!h]
%	\center{\includegraphics[scale=0.76]{pics/13_1.jpg}}
%\end{figure}
%
%\clearpage
%\begin{figure}[!h]
%	\center{\includegraphics[scale=0.76]{pics/13_2.jpg}}
%\end{figure}

\end{document}