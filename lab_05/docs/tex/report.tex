\documentclass[12pt]{report}
\usepackage[utf8]{inputenc}
\usepackage[russian]{babel}
%\usepackage[14pt]{extsizes}
\usepackage{listings}
\usepackage{graphicx}
\usepackage{amsmath,amsfonts,amssymb,amsthm,mathtools} 
\usepackage{pgfplots}
\usepackage{filecontents}
\usepackage{indentfirst}
\usepackage{eucal}
\usepackage{float} 
\usepackage{amsmath}
\usepackage{enumitem}
\usepackage[justification=centering]{caption} 
\usepackage{tikz}
\usepackage{pgfplots}
\pgfplotsset{compat=newest}

\frenchspacing

\usepackage{indentfirst} % Красная строка


%\usetikzlibrary{datavisualization}
%\usetikzlibrary{datavisualization.formats.functions}

\usepackage{amsmath}


% Для листинга кода:
\lstset{ %
	language=haskell,                 % выбор языка для подсветки (здесь это С)
	basicstyle=\small\sffamily, % размер и начертание шрифта для подсветки кода
	numbers=left,               % где поставить нумерацию строк (слева\справа)
	numberstyle=\tiny,           % размер шрифта для номеров строк
	stepnumber=1,                   % размер шага между двумя номерами строк
	numbersep=5pt,                % как далеко отстоят номера строк от подсвечиваемого кода
	showspaces=false,            % показывать или нет пробелы специальными отступами
	showstringspaces=false,      % показывать или нет пробелы в строках
	showtabs=false,             % показывать или нет табуляцию в строках
	frame=single,              % рисовать рамку вокруг кода
	tabsize=2,                 % размер табуляции по умолчанию равен 2 пробелам
	captionpos=t,              % позиция заголовка вверху [t] или внизу [b] 
	breaklines=true,           % автоматически переносить строки (да\нет)
	breakatwhitespace=false, % переносить строки только если есть пробел
	escapeinside={\#*}{*)}   % если нужно добавить комментарии в коде
}

\usepackage[left=2cm,right=2cm, top=2cm,bottom=2cm,bindingoffset=0cm]{geometry}

\usepackage{listings}

\usepackage{titlesec}
\titleformat{\section}
{\normalsize\bfseries}
{\thesection}
{1em}{}
\titlespacing*{\chapter}{0pt}{-30pt}{8pt}
\titlespacing*{\section}{\parindent}{*4}{*4}
\titlespacing*{\subsection}{\parindent}{*4}{*4}
\usepackage{setspace}

\titleformat{\chapter}{\LARGE\bfseries}{\thechapter}{20pt}{\large\bfseries}
\titleformat{\section}{\Large\bfseries}{\thesection}{20pt}{\large\bfseries}

\makeatletter 

\begin{document}
	
%\def\chaptername{} % убирает "Глава"
\thispagestyle{empty}
\begin{titlepage}
	\noindent \begin{minipage}{0.15\textwidth}
		\includegraphics[width=\linewidth]{pics/logo}
	\end{minipage}
	\noindent\begin{minipage}{0.9\textwidth}\centering
		\textbf{Министерство науки и высшего образования Российской Федерации}\\
		\textbf{Федеральное государственное бюджетное образовательное учреждение высшего образования}\\
		\textbf{~~~«Московский государственный технический университет имени Н.Э.~Баумана}\\
		\textbf{(национальный исследовательский университет)»}\\
		\textbf{(МГТУ им. Н.Э.~Баумана)}
	\end{minipage}
	
	\noindent\rule{18cm}{3pt}
	\newline\newline
	\noindent ФАКУЛЬТЕТ $\underline{\text{«Информатика и системы управления»}}$ \newline\newline
	\noindent КАФЕДРА $\underline{\text{«Программное обеспечение ЭВМ и информационные технологии»}}$\newline\newline\newline\newline\newline
	
	
	\begin{center}
		\noindent\begin{minipage}{1.3\textwidth}\centering
			\Large\textbf{  Отчёт по лабораторной работе №5 по курсу}\newline\newline
			\textbf{<<Функциональное и логическое}\newline
			\textbf{\indent\indent\indent программирование>>}\newline
		\end{minipage}
	\end{center}
	
	~\\\\\\\\\\\\
	\large
	\noindent\textbf{Тема } $\underline{\text{Использование управляющих структур, работа со списками.}}$\newline\newline
	\noindent\textbf{Студент } $\underline{\text{Сироткина П.Ю.}}$\newline\newline
	\noindent\textbf{Группа } $\underline{\text{ИУ7-66Б}}$\newline\newline
	\noindent\textbf{Преподаватели } $\underline{\text{Толпинская Н.Б., Строганов Ю.В.}}$\newline\newline\newline
	
	\begin{center}
		\vfill
		Москва~---~\the\year
		~г.
	\end{center}
\end{titlepage}

\chapter*{\Large Практические задания}

~\
%\addcontentsline{toc}{chapter}{Введение}

\section*{1. Написать функцию, которая по своему списку-аргументу lst определяет, является ли он палиндромом (то есть равны ли lst и reverse(lst)).}

\begin{lstlisting}
	(defun palindrome(lst) (equal lst (reverse lst)))
\end{lstlisting}

\section*{2. Написать предикат set-equal, который возвращает T, если два его множества-аргумента содержат одни и те же элементы, порядок которых не имеет значения.}

\begin{lstlisting}
	(defun set_equal(set1 set2)
		(and (= (length set1) (length set2))
			 (every #'(lambda (elem) (member elem set1 :test #'equal)) set2)
			 (every #'(lambda (elem) (member elem set2 :test #'equal)) set1)))
\end{lstlisting}

\section*{3. Напишите свои необходимые функции, которые обрабатывают таблицу из 4-х точечных пар: (страна.столица), и возвращает по стране столицу, а по столице - страну.}

\begin{lstlisting}
	(defun get_country(table capital)
			(cond ((null table) Nil)
						((equal (cdar table) capital) (caar table))
						(T (get_country (cdr table) capital))))
	
	(defun get_country(table capital)
			(car (rassoc capital table)))
	
	(defun get_capital(table country)
			(cond ((null table) Nil)
						((equal (caar table) country) (cdar table))
						(T (get_capital (cdr table) country))))
	
	(defun get_capital(table country)
			(cdr (assoc country table)))
\end{lstlisting}

\clearpage

\section*{4. Напишите функцию swap-first-last, которая переставляет в списке-аргументе первый и последний элементы.}

\begin{lstlisting}
	(defun swap_first_last(lst)
		(let ((first_elem (car lst))) 
					(setf (car lst) (car (last lst)))
					(setf (car (last lst)) first_elem)
					lst))
\end{lstlisting}

\section*{5. Напишите функцию swap-two-element, которая переставляет в списке-аргументе два указанных своими порядковыми номерами элемента в этом списке.}

\begin{lstlisting}
	(defun swap_two_elements(lst i j)
			(if (and (< -1 i (length lst)) (< -1 j (length lst)))
					(let ((i_elem (nth i lst)) (j_elem (nth j lst)))
							(setf (nth i lst) j_elem) (setf (nth j lst) i_elem) lst)
					lst))
\end{lstlisting}

\section*{6. Напишите две функции, swap-to-left и swap-to-right, которые производят одну круговую перестановку в списке-аргументе влево и вправо соответственно.}

\begin{lstlisting}
	(defun swap_to_left(lst)
		(cond ((null lst) Nil)
				(T (append (cdr lst) (cons (car lst) Nil)))))
	
	(defun swap_to_right(lst)
		(cond ((null lst) Nil)
				(T (cons (car (reverse lst)) (reverse (cdr (reverse lst)))))))
\end{lstlisting}

\section*{7. Напишите функцию, которая добавляет к множеству двухэлементных списков новый двухэлементный список, если его там нет.}

\begin{lstlisting}
	(defun add_pair_to_set(src_set pair)
		(if (not (member pair src_set :test #'equal)) 
							(append src_set (list pair))
							 src_set))
\end{lstlisting}

\section*{8. Напишите функцию, которая умножает на заданное число-аргумент первый числовой элемент списка из заданного 3-х элементного списка-аргумента, когда}

\begin{itemize}
	\item Все элементы списка -- числа;
	\item Элементы списка -- любые объекта.
\end{itemize}

\begin{lstlisting}
	(defun mul_num(lst multiplier)
	(mapcar #'(lambda(elem) (* elem multiplier)) lst))
	
	(defun mul(lst multiplier)
		(mapcar #'(lambda (elem)
													  (cond ((listp elem) (mul (cdr elem) multiplier))
																	((numberp elem) (* elem multiplier))
																	(T elem))) lst))
\end{lstlisting}

\section*{9. Напишите функцию select-between, которая из списка-аргумента из 5 чисел выбирает только те, которые расположены между двумя указанными границами-аргументами и возвращает их в виде списка (упорядоченного по возрастанию списка чисел (+2 балла)).}

\begin{lstlisting}
	(defun select_between(lst a b)
			(cond ((null lst) Nil)
				   ((< a (car lst) b)(cons (car lst)(select_between (cdr lst) a b)))
				   (T (select_between (cdr lst) a b))))
\end{lstlisting}

\end{document}