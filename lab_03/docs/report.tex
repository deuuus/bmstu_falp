\documentclass[12pt]{report}
\usepackage[utf8]{inputenc}
\usepackage[russian]{babel}
%\usepackage[14pt]{extsizes}
\usepackage{listings}
\usepackage{graphicx}
\usepackage{amsmath,amsfonts,amssymb,amsthm,mathtools} 
\usepackage{pgfplots}
\usepackage{filecontents}
\usepackage{indentfirst}
\usepackage{eucal}
\usepackage{float} 
\usepackage{amsmath}
\usepackage{enumitem}
\usepackage[justification=centering]{caption} 
\usepackage{tikz}
\usepackage{pgfplots}
\pgfplotsset{compat=newest}

\frenchspacing

\usepackage{indentfirst} % Красная строка


%\usetikzlibrary{datavisualization}
%\usetikzlibrary{datavisualization.formats.functions}

\usepackage{amsmath}


% Для листинга кода:
\lstset{ %
	language=haskell,                 % выбор языка для подсветки (здесь это С)
	basicstyle=\small\sffamily, % размер и начертание шрифта для подсветки кода
	numbers=left,               % где поставить нумерацию строк (слева\справа)
	numberstyle=\tiny,           % размер шрифта для номеров строк
	stepnumber=1,                   % размер шага между двумя номерами строк
	numbersep=5pt,                % как далеко отстоят номера строк от подсвечиваемого кода
	showspaces=false,            % показывать или нет пробелы специальными отступами
	showstringspaces=false,      % показывать или нет пробелы в строках
	showtabs=false,             % показывать или нет табуляцию в строках
	frame=single,              % рисовать рамку вокруг кода
	tabsize=2,                 % размер табуляции по умолчанию равен 2 пробелам
	captionpos=t,              % позиция заголовка вверху [t] или внизу [b] 
	breaklines=true,           % автоматически переносить строки (да\нет)
	breakatwhitespace=false, % переносить строки только если есть пробел
	escapeinside={\#*}{*)}   % если нужно добавить комментарии в коде
}

\usepackage[left=2cm,right=2cm, top=2cm,bottom=2cm,bindingoffset=0cm]{geometry}

\usepackage{listings}

\usepackage{titlesec}
\titleformat{\section}
{\normalsize\bfseries}
{\thesection}
{1em}{}
\titlespacing*{\chapter}{0pt}{-30pt}{8pt}
\titlespacing*{\section}{\parindent}{*4}{*4}
\titlespacing*{\subsection}{\parindent}{*4}{*4}
\usepackage{setspace}

\titleformat{\chapter}{\LARGE\bfseries}{\thechapter}{20pt}{\large\bfseries}
\titleformat{\section}{\Large\bfseries}{\thesection}{20pt}{\large\bfseries}

\makeatletter 

\begin{document}
	
%\def\chaptername{} % убирает "Глава"
\thispagestyle{empty}
\begin{titlepage}
	\noindent \begin{minipage}{0.15\textwidth}
		\includegraphics[width=\linewidth]{pics/logo}
	\end{minipage}
	\noindent\begin{minipage}{0.9\textwidth}\centering
		\textbf{Министерство науки и высшего образования Российской Федерации}\\
		\textbf{Федеральное государственное бюджетное образовательное учреждение высшего образования}\\
		\textbf{~~~«Московский государственный технический университет имени Н.Э.~Баумана}\\
		\textbf{(национальный исследовательский университет)»}\\
		\textbf{(МГТУ им. Н.Э.~Баумана)}
	\end{minipage}
	
	\noindent\rule{18cm}{3pt}
	\newline\newline
	\noindent ФАКУЛЬТЕТ $\underline{\text{«Информатика и системы управления»}}$ \newline\newline
	\noindent КАФЕДРА $\underline{\text{«Программное обеспечение ЭВМ и информационные технологии»}}$\newline\newline\newline\newline\newline
	
	
	\begin{center}
		\noindent\begin{minipage}{1.3\textwidth}\centering
			\Large\textbf{  Отчёт по лабораторной работе №3 по курсу}\newline\newline
			\textbf{<<Функциональное и логическое}\newline
			\textbf{\indent\indent\indent программирование>>}\newline
		\end{minipage}
	\end{center}
	
	~\\\\\\\\\\\\
	\large
	\noindent\textbf{Тема } $\underline{\text{Работа интерпретатора Lisp.}}$\newline\newline
	\noindent\textbf{Студент } $\underline{\text{Сироткина П.Ю.}}$\newline\newline
	\noindent\textbf{Группа } $\underline{\text{ИУ7-66Б}}$\newline\newline
	\noindent\textbf{Преподаватели } $\underline{\text{Толпинская Н.Б., Строганов Ю.В.}}$\newline\newline\newline
	
	\begin{center}
		\vfill
		Москва~---~\the\year
		~г.
	\end{center}
\end{titlepage}

\chapter{Практические задания}
%\addcontentsline{toc}{chapter}{Введение}

\section{Написать функцию, которая принимает целое число и возвращает первое четное число, не меньшее аргумента.}

\begin{lstlisting}
	(defun make_even(x) (if (evenp x) x (+ x 1)))
\end{lstlisting}

\section{Написать функцию, которая принимает число и возвращает число того же знака, но с модулем на 1 больше модуля аргумента.}

\begin{lstlisting}
	(defun inc_abs(x) (if (> x 0) (+ x 1) (- x 1)))
\end{lstlisting}

\section{Написать функцию, которая принимает два числа и возвращает список из этих чисел, расположенный по возрастанию.}

\begin{lstlisting}
	(defun asc_pair(x y) (if (> x y) (list y x) (list x y)))
\end{lstlisting}
	
\section{Написать функцию, которая принимает 3 числа и возвращает T только тогда, когда первое число расположено между вторым и третьим.}	

\begin{lstlisting}
	(defun f_btw_st(x y z) (< y x z))
\end{lstlisting}

\section{Каков результат вычисления следующих выражений?}

\begin{lstlisting}
	(and 'fee 'fie 'foe)          ; FOE
	(or Nil 'fie 'foe)            ; FIE
	(and (equal 'abc 'abc) 'yes)  ; YES 
	(or 'fee 'fie 'foe)           ; FEE
	(and Nil 'fie 'foe)           ; NIL
	(or (equal 'abc 'abc) 'yes)   ; T
\end{lstlisting}

\section{Написать предикат, который принимает два числа-аргумента и возвращает T, если первое число не меньше второго.}

\begin{lstlisting}
	(defun eg(x y) (>= x y))
\end{lstlisting}

\section{Какой из следующих двух вариантов предиката ошибочный и почему?}

\begin{lstlisting}
	(defun pred1(x) (and (numberp x) (plusp x))) ; OK
	(defun pred2(x) (and (plusp x) (numberp x))) ; ERROR
\end{lstlisting}

Второй вариант предиката ошибочен, потому что аргумент предиката $PLUSP$ обязан быть числом, иначе возникнет ошибка, поэтому сначала необходимо использовать предикат $NUMBERP$, чтобы определить, является ли аргумент числом.

\section{Решить задачу 4, используя для ее решения конструкции IF, COND, AND/OR.}

\begin{lstlisting}
	(defun if_btw(x y z) 
						(if (< y x z) T Nil))
\end{lstlisting}
\begin{lstlisting}	
	(defun cond_btw(x y z) 
						(cond ((and (> x y) (< x z)) T) 
							  	(T Nil))
\end{lstlisting}
\begin{lstlisting}
	(defun and_btw(x y z) 
						(and (> x y) (< x z)))
\end{lstlisting}

\section{Переписать функцию how-alike, приведенную в лекции и использующую COND, используя только конструкции IF, COND, AND/OR.}

\begin{lstlisting}
	(defun if_how_alike(x y)
		(if (= x y) 'the_same 
			(if (if (oddp x) (oddp y)) 'both_odd
				(if (if (evenp x) (evenp y)) 'both_even 'diff))))
\end{lstlisting}
\begin{lstlisting}
	(defun cond_how_alike(x y)
		(cond (= x y) 'the_same)
		((and (oddp x) (oddp y)) 'both_odd)
		((and (evenp x) (evenp y)) 'both_even)
		(T 'diff)
	)
\end{lstlisting}
\begin{lstlisting}
	(defun and-or_how_alike(x y)
		(or
			(and (= x y) 'the_same)
			(and (oddp x) (oddp y) 'both_odd)
			(and (evenp x) (evenp y) 'both_even)
			'diff
		)
	)
\end{lstlisting}

\end{document}