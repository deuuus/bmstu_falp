\documentclass[12pt]{report}
\usepackage[utf8]{inputenc}
\usepackage[russian]{babel}
%\usepackage[14pt]{extsizes}
\usepackage{listings}
\usepackage{graphicx}
\usepackage{amsmath,amsfonts,amssymb,amsthm,mathtools} 
\usepackage{pgfplots}
\usepackage{filecontents}
\usepackage{indentfirst}
\usepackage{eucal}
\usepackage{float} 
\usepackage{amsmath}
\usepackage{enumitem}
\usepackage[justification=centering]{caption} 
\usepackage{tikz}
\usepackage{pgfplots}
\pgfplotsset{compat=newest}

\frenchspacing

\usepackage{indentfirst} % Красная строка


%\usetikzlibrary{datavisualization}
%\usetikzlibrary{datavisualization.formats.functions}

\usepackage{amsmath}


% Для листинга кода:
\lstset{ %
	language=haskell,                 % выбор языка для подсветки (здесь это С)
	basicstyle=\small\sffamily, % размер и начертание шрифта для подсветки кода
	numbers=left,               % где поставить нумерацию строк (слева\справа)
	numberstyle=\tiny,           % размер шрифта для номеров строк
	stepnumber=1,                   % размер шага между двумя номерами строк
	numbersep=5pt,                % как далеко отстоят номера строк от подсвечиваемого кода
	showspaces=false,            % показывать или нет пробелы специальными отступами
	showstringspaces=false,      % показывать или нет пробелы в строках
	showtabs=false,             % показывать или нет табуляцию в строках
	frame=single,              % рисовать рамку вокруг кода
	tabsize=2,                 % размер табуляции по умолчанию равен 2 пробелам
	captionpos=t,              % позиция заголовка вверху [t] или внизу [b] 
	breaklines=true,           % автоматически переносить строки (да\нет)
	breakatwhitespace=false, % переносить строки только если есть пробел
	escapeinside={\#*}{*)}   % если нужно добавить комментарии в коде
}

\usepackage[left=2cm,right=2cm, top=2cm,bottom=2cm,bindingoffset=0cm]{geometry}

\usepackage{listings}

\usepackage{titlesec}
\titleformat{\section}
{\normalsize\bfseries}
{\thesection}
{1em}{}
\titlespacing*{\chapter}{0pt}{-30pt}{8pt}
\titlespacing*{\section}{\parindent}{*4}{*4}
\titlespacing*{\subsection}{\parindent}{*4}{*4}
\usepackage{setspace}

\titleformat{\chapter}{\LARGE\bfseries}{\thechapter}{20pt}{\large\bfseries}
\titleformat{\section}{\Large\bfseries}{\thesection}{20pt}{\large\bfseries}

\makeatletter 

\begin{document}
	
%\def\chaptername{} % убирает "Глава"
\thispagestyle{empty}
\begin{titlepage}
	\noindent \begin{minipage}{0.15\textwidth}
		\includegraphics[width=\linewidth]{pics/logo}
	\end{minipage}
	\noindent\begin{minipage}{0.9\textwidth}\centering
		\textbf{Министерство науки и высшего образования Российской Федерации}\\
		\textbf{Федеральное государственное бюджетное образовательное учреждение высшего образования}\\
		\textbf{~~~«Московский государственный технический университет имени Н.Э.~Баумана}\\
		\textbf{(национальный исследовательский университет)»}\\
		\textbf{(МГТУ им. Н.Э.~Баумана)}
	\end{minipage}
	
	\noindent\rule{18cm}{3pt}
	\newline\newline
	\noindent ФАКУЛЬТЕТ $\underline{\text{«Информатика и системы управления»}}$ \newline\newline
	\noindent КАФЕДРА $\underline{\text{«Программное обеспечение ЭВМ и информационные технологии»}}$\newline\newline\newline\newline\newline
	
	
	\begin{center}
		\noindent\begin{minipage}{1.3\textwidth}\centering
			\Large\textbf{  Отчёт по лабораторной работе №6 по курсу}\newline\newline
			\textbf{<<Функциональное и логическое}\newline
			\textbf{\indent\indent\indent программирование>>}\newline
		\end{minipage}
	\end{center}
	
	~\\\\\\\\\\\\
	\large
	\noindent\textbf{Тема } $\underline{\text{Использование функционалов.}}$\newline\newline
	\noindent\textbf{Студент } $\underline{\text{Сироткина П.Ю.}}$\newline\newline
	\noindent\textbf{Группа } $\underline{\text{ИУ7-66Б}}$\newline\newline
	\noindent\textbf{Преподаватели } $\underline{\text{Толпинская Н.Б., Строганов Ю.В.}}$\newline\newline\newline
	
	\begin{center}
		\vfill
		Москва~---~\the\year
		~г.
	\end{center}
\end{titlepage}

\chapter*{Практические задания}
%\addcontentsline{toc}{chapter}{Введение}

~\

\section*{1. Напишите функцию, которая уменьшает на 10 все числа из списка-аргумента этой функции.}

\begin{lstlisting}
	(defun dec_10(lst) 
				(mapcar #'(lambda (elem) 
											(cond ((numberp elem) (- elem 10)) (T elem))) lst))
\end{lstlisting}

\section*{2. Напишите функцию, которая умножает на заданное число-аргумент все числа из заданного списка-аргумента, когда:}

\begin{enumerate}
	\item Все элементы списка -- числа;
	\item Элементы списка -- любые объекты.
\end{enumerate}

\begin{lstlisting}
	(defun mul_num(lst multiplier)
			(mapcar #'(lambda(elem) (* elem multiplier)) lst))
	
	(defun mul(lst multiplier)
			(mapcar #'(lambda (elem)
														(cond ((listp elem) (mul (cdr elem) multiplier))
														((numberp elem) (* elem multiplier))
														(T elem))) lst))
\end{lstlisting}

\section*{3. Написать функцию, которая по своему списку-аргументу lst определяет, является ли он палиндромом (то есть равны ли lst и reverse(lst)).}

\begin{lstlisting}
	(defun my_reverse(lst)
			(reduce #'(lambda (res lst) (cons lst res)) lst :initial-value ()))
	
	(defun palindrome(lst)
			(equal lst (my_reverse lst)))
\end{lstlisting}

\clearpage
\section*{4. Написать предикат set-equal, который возвращает T, если два его множества-аргумента содержат одни и те же элементы, порядок которых не имеет значения.}

\begin{lstlisting}
	(defun in_set(elem src_set) 
				(reduce #'(lambda (x y) (or x y)) 
						(mapcar #'(lambda (x) (equal x elem)) src_set) 
						:initial-value Nil))
	
	(defun is_subset(set1 set2)
				(reduce #'(lambda (x y) (and x y)) 
						(mapcar #'(lambda (x) (in_set x set2)) set1) :initial-value T))
	
	(defun set_equal(set1 set2)
				(and (is_subset set1 set2) (is_subset set2 set1)))
\end{lstlisting}

\section*{5. Написать функцию, которая получает как аргумент список чисел, а возвращает список квадратов этих чисел в том же порядке.}

\begin{lstlisting}
	(defun make_squares(lst) (mapcar #'(lambda (elem) (* elem elem)) lst))
\end{lstlisting}

\section*{6. Написать функцию select-between, которая из списка-аргумента, содержащего только числа, выбирает только те, которые расположены между двумя указанными границами-аргументами и возвращает их в виде списка (упорядоченного по возрастанию списка чисел (+ 2 балла)).}

\begin{lstlisting}
	(defun select_between(lst left right)
			(sort (reduce #'(lambda (res_lst elem) 
								(if (< left elem right) (cons elem res_lst) res_lst)) 
												lst :initial-value ()) #'<))
\end{lstlisting}

\section*{7. Написать функцию, вычисляющую декартово произведение двух своих списков-аргументов.}

\begin{lstlisting}
(defun decart(lst_x lst_y)
		   (mapcan #'(lambda(x) (mapcar #'(lambda(y) (list x y)) lst_y)) lst_x))
\end{lstlisting}

\section*{8. Почему так реализовано reduce, в чем причина?}

\begin{lstlisting}
	(reduce #'+ 0) -> 0
	(reduce #'+ ()) -> 0
\end{lstlisting}

Начальный результат работы функционала \textbf{reduce} - т.н. нейтральный. Для операции сумма нейтральным элементом является 0. 

\section*{9. Пусть list-of-list список, состоящий из списков. Написать функцию, которая вычисляет сумму длин всех элементов list-of-list.}

\begin{lstlisting}
(defun sum_lengths(lol) 
    (reduce #'(lambda(cur_sum lst) (+ cur_sum (length lst))) (cons 0 lol)))
\end{lstlisting}

\end{document}