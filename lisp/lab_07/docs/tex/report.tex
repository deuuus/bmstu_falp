\documentclass[12pt]{report}
\usepackage[utf8]{inputenc}
\usepackage[russian]{babel}
%\usepackage[14pt]{extsizes}
\usepackage{listings}
\usepackage{graphicx}
\usepackage{amsmath,amsfonts,amssymb,amsthm,mathtools} 
\usepackage{pgfplots}
\usepackage{filecontents}
\usepackage{indentfirst}
\usepackage{eucal}
\usepackage{float} 
\usepackage{amsmath}
\usepackage{enumitem}
\usepackage[justification=centering]{caption} 
\usepackage{tikz}
\usepackage{pgfplots}
\pgfplotsset{compat=newest}

\frenchspacing

\usepackage{indentfirst} % Красная строка


%\usetikzlibrary{datavisualization}
%\usetikzlibrary{datavisualization.formats.functions}

\usepackage{amsmath}


% Для листинга кода:
\lstset{ %
	language=haskell,                 % выбор языка для подсветки (здесь это С)
	basicstyle=\small\sffamily, % размер и начертание шрифта для подсветки кода
	numbers=left,               % где поставить нумерацию строк (слева\справа)
	numberstyle=\tiny,           % размер шрифта для номеров строк
	stepnumber=1,                   % размер шага между двумя номерами строк
	numbersep=5pt,                % как далеко отстоят номера строк от подсвечиваемого кода
	showspaces=false,            % показывать или нет пробелы специальными отступами
	showstringspaces=false,      % показывать или нет пробелы в строках
	showtabs=false,             % показывать или нет табуляцию в строках
	frame=single,              % рисовать рамку вокруг кода
	tabsize=2,                 % размер табуляции по умолчанию равен 2 пробелам
	captionpos=t,              % позиция заголовка вверху [t] или внизу [b] 
	breaklines=true,           % автоматически переносить строки (да\нет)
	breakatwhitespace=false, % переносить строки только если есть пробел
	escapeinside={\#*}{*)}   % если нужно добавить комментарии в коде
}

\usepackage[left=2cm,right=2cm, top=2cm,bottom=2cm,bindingoffset=0cm]{geometry}

\usepackage{listings}

\usepackage{titlesec}
\titleformat{\section}
{\normalsize\bfseries}
{\thesection}
{1em}{}
\titlespacing*{\chapter}{0pt}{-30pt}{8pt}
\titlespacing*{\section}{\parindent}{*4}{*4}
\titlespacing*{\subsection}{\parindent}{*4}{*4}
\usepackage{setspace}

\titleformat{\chapter}{\LARGE\bfseries}{\thechapter}{20pt}{\large\bfseries}
\titleformat{\section}{\Large\bfseries}{\thesection}{20pt}{\large\bfseries}

\makeatletter 

\begin{document}
	
%\def\chaptername{} % убирает "Глава"
\thispagestyle{empty}
\begin{titlepage}
	\noindent \begin{minipage}{0.15\textwidth}
		\includegraphics[width=\linewidth]{pics/logo}
	\end{minipage}
	\noindent\begin{minipage}{0.9\textwidth}\centering
		\textbf{Министерство науки и высшего образования Российской Федерации}\\
		\textbf{Федеральное государственное бюджетное образовательное учреждение высшего образования}\\
		\textbf{~~~«Московский государственный технический университет имени Н.Э.~Баумана}\\
		\textbf{(национальный исследовательский университет)»}\\
		\textbf{(МГТУ им. Н.Э.~Баумана)}
	\end{minipage}
	
	\noindent\rule{18cm}{3pt}
	\newline\newline
	\noindent ФАКУЛЬТЕТ $\underline{\text{«Информатика и системы управления»}}$ \newline\newline
	\noindent КАФЕДРА $\underline{\text{«Программное обеспечение ЭВМ и информационные технологии»}}$\newline\newline\newline\newline\newline
	
	
	\begin{center}
		\noindent\begin{minipage}{1.3\textwidth}\centering
			\Large\textbf{  Отчёт по лабораторной работе №7 по курсу}\newline\newline
			\textbf{<<Функциональное и логическое}\newline
			\textbf{\indent\indent\indent программирование>>}\newline
		\end{minipage}
	\end{center}
	
	~\\\\\\\\\\\\
	\large
	\noindent\textbf{Тема } $\underline{\text{Рекурсивные функции.}}$\newline\newline
	\noindent\textbf{Студент } $\underline{\text{Сироткина П.Ю.}}$\newline\newline
	\noindent\textbf{Группа } $\underline{\text{ИУ7-66Б}}$\newline\newline
	\noindent\textbf{Преподаватели } $\underline{\text{Толпинская Н.Б., Строганов Ю.В.}}$\newline\newline\newline
	
	\begin{center}
		\vfill
		Москва~---~\the\year
		~г.
	\end{center}
\end{titlepage}

\chapter*{Практические задания}
%\addcontentsline{toc}{chapter}{Введение}

~\

\section*{1. Написать хвостовую рекурсивную функцию my-reverse, которая развернет верхний уровень своего списка-аргумента lst.}

\begin{lstlisting}
	(defun my_reverse(lst)
			(move_to lst Nil)
	)
	
	(defun move_to(lst result)
			(cond
					((null lst) result)
					(T (move_to (cdr lst) (cons (car lst) result)))
			)
	)
\end{lstlisting}

\section*{2. Написать функцию, которая возвращает первый элемент списка-аргумента, который сам является непустым списком.}

\begin{lstlisting}
	(defun first_sublist(lst)
			(cond 
					((null lst) Nil)
					((atom (car lst)) (first_sublist (cdr lst)))
					(T (caar lst))
			)
	)
\end{lstlisting}

\section*{3. Написать функцию, которая выбирает из заданного списка только те числа, которые больше 1 и меньше 10.}

\begin{lstlisting}
	(defun select_between(lst)
			(cond
					((null lst) Nil)
					((< 1 (car lst) 10) (cons (car lst) (select_between (cdr lst))))
					(T (select_between (cdr lst)))
			)
	)
\end{lstlisting}

\section*{4. Напишите рекурсивную функцию, которая умножает на заданное число-аргумент все числа из заданного списка-аргумента, когда: }

\begin{enumerate}
	\item Все элементы списка -- числа;
	\item Элементы списка - любые объекты.
\end{enumerate}

\begin{lstlisting}
(defun mul_num(lst mul)
	(cond 
			((null lst) Nil)
			(T (cons (* (car lst) mul) (mul_num (cdr lst) mul)))
	)
)

(defun mul_all(lst mul)
	(cond 
			((null lst) Nil)
			((numberp (car lst)) (cons (* (car lst) mul) (mul_all(cdr lst) mul)))
			((atom (car lst)) (cons (car lst) (mul_all (cdr lst) mul)))
			(T (cons (mul_all (car lst) mul) (mul_all (cdr lst) mul)))
	)
)
\end{lstlisting}

\section*{5. Напишите функцию select-between, которая из списка-аргумента, содержащего только числа, выбирает только те, которые расположены между двумя указанными границами-аргументами и возвращает их в виде списка.}

\begin{lstlisting}
(defun select_between(lst left right)
		(cond
				((null lst) Nil)
				((< left (car lst) right) 
									(cons (car lst) (select_between (cdr lst) left right)))
				(T (select_between (cdr lst) left right))
		)
)
\end{lstlisting}

\clearpage
\section*{6. Написать рекурсивную версию (с именем rec-add) вычисления суммы чисел заданного списка:}

\begin{enumerate}
	\item Одноуровнего смешанного;
	\item Структурированного.
\end{enumerate}

\begin{lstlisting}
;1
(defun rec_add(lst) (helper lst 0))

(defun helper(lst acc)
		(cond
				((null lst) acc)
				((numberp (car lst)) (helper (cdr lst) (+ acc (car lst))))
				(T (helper (cdr lst) acc))
		)
)

;2
(defun rec_add(lst) (helper lst 0))

(defun helper(lst acc)
		(cond 
				((null lst) acc)
				((numberp (car lst)) (helper (cdr lst) (+ acc (car lst))))
				((listp (car lst)) (+ (helper (car lst) 0) (helper (cdr lst) acc)))
				(T (helper (cdr lst) acc))
		)
)
\end{lstlisting}

\section*{7. Написать рекурсивную версию с именем recnth функции nth.}

\begin{lstlisting}
	(defun recnth(n lst)
			(cond 
					((null lst) Nil)
					((= 0 n) (car lst))
					(T (recnth (- n 1) (cdr lst)))))
\end{lstlisting}


\section*{8. Написать рекурсивную функцию allodd, которая возвращает t, когда все элементы списка нечетные.}

\begin{lstlisting}
	(defun allodd(lst)
			(cond
					((null lst) T)
					((oddp (car lst)) (allodd (cdr lst)))
					(T Nil)
			)
	)
\end{lstlisting}

\section*{9. Написать рекурсивную функцию, которая возвращает первое нечетное число из списка (структурированного), возможно создавая некоторые вспомогательные функции.}

\begin{lstlisting}
	(defun first_atom(lst)
			(cond
					((atom lst) lst)
					(T (first_atom (car lst)))
			)
	)   
	
	(defun first_odd(lst)
			(cond
					((null lst) Nil)
					((oddp (first_atom lst)) (first_atom lst))
					(T (first_odd (cdr lst)))
			)
	)
\end{lstlisting}

\section*{10. Используя cons-дополняемую рекурсию с одним тестом завершения, написать функцию, которая получает как аргумент список чисел, а возвращает список квадратов этих чисел в том же порядке.}

\begin{lstlisting}
	(defun squares(lst)
			(cond
					((null lst) Nil)
					(T (cons ((lambda (x) (* x x)) (car lst)) (squares (cdr lst))))
			)
	)
\end{lstlisting}

\end{document}